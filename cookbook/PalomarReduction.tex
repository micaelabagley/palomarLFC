\documentclass{article}
\usepackage{graphicx}
\usepackage{vmargin}
\usepackage{hyperref}
\usepackage{amsmath}
\usepackage{url}
\setpapersize{A4}
\setmarginsrb{2.5cm}{2.5cm}{2.4cm}{2.8cm}{12pt}{14pt}{12pt}{22pt}
\pagestyle{empty}

%%% define \maketitle without the vertical space %%%
\makeatletter
\def\@maketitle{%
  \newpage
%  \null% DELETED
%  \vskip 2em% DELETED
  \begin{center}%
  \let \footnote \thanks
    {\LARGE \@title \par}%
    \vskip 0.75em%
    {\large
      \lineskip .5em%
      \begin{tabular}[t]{c}%
        \@author
      \end{tabular}\par}%
    \vskip 0.75em%
    {\small \@date}%
  \end{center}%
  \par
  \vskip 1.5em}
\makeatother
%%% reduce spacing between items in lists %%%
\newlength{\wideitemsep}
\setlength{\wideitemsep}{.5\itemsep}
\addtolength{\wideitemsep}{-7pt}
\let\olditem\item
\renewcommand{\item}{\setlength{\itemsep}{\wideitemsep}\olditem}
%%% reduce size of $\sim$ for use with bash paths %%%
\newcommand{\ttsim}{\raise.17ex\hbox{$\scriptstyle\mathtt{\sim}$}}
%%% define how bash code will be displayed (indented, preceded by '#' %%%
\newcommand{\shellcmd}[1]{\indent\indent\texttt{\small\# #1}\\}
%%%
\newcommand{\An}{\textit{Astrometry.net}}

\title{Palomar 200" LFC Reduction Cookbook}
\author{Micaela Bagley}
\date{December 2014}

\begin{document}
\maketitle

\section{Reduction}
\subsection{Organize the data}
The first step is to sort the data from a night of observations. 
The \texttt{extract\_header} module in \texttt{sort\_images.py} will
produce a file listing all images in a directory as well as their
image type, object name, exposure time, filter, airmass, and binning 
($1\times1$ or $2\times2$).\\

\texttt{usage: sort\_images.py [-h] [--extract\_header] directory} \\

\texttt{directory} $-$ \hangindent=2.7cm the directory for which to extract
information from all image headers.\\

\noindent You should have a directory for each night of science data (e.g.
\texttt{23mar12/}) and a directory for each of the types of calibration 
images to be used in reducing the science data (a directory each for 
biases, flats, and if necessary, darks). The directories for calibration 
images may be wherever you like (inside the
directory of science images, some random place on your computer, etc.)
as long as the calibration files have their own directories. \\

\noindent Inspect all calibration images and put any biases you don't 
want to use inside a directory called \texttt{unused/} within the biases 
directory. The same goes for the flats (and darks). The naming convention
for directories is necessary if you wish the reduction pipeline to 
automatically produce a log file. 

\vspace{3 mm}
\section{Astrometry \& Alignment}
WCS 

After reducing the data, sort all images by WISP parallel field with
one field per directory. Output files will be named after the directory.
For example, images in \texttt{Par300/} will be aligned and combined
to form \texttt{Par300\_[filter].fits}.
There are several fields that were observed on multiple nights. To avoid 
overwriting files, give each FITS file a unique name 
(include the date observed, for example). \\

\noindent Astrometry and image alignment is performed in $4$ steps:
\begin{enumerate}
\item \texttt{astrometry.py}
\item \texttt{fix\_astrometry.py}
\item \texttt{combine.py}
\item \texttt{align\_images.py} ~~(\texttt{imalign.pro})
\end{enumerate}
\subsection{\texttt{astrometry.py}}
The first step is to get a rough (though very good) approximate WCS 
solution for each image in a WISP parallel field using the \An~software 
package. \texttt{astrometry.py} builds the command that will
run the \An~software on each image in a directory. \\

\texttt{usage: astrometry.py [-h] [--useSE] wispfield} \\

\texttt{wispfield} $-$ \hangindent=2.7cm the WISP parallel field whose images
are to be fit with WCS solutions. This must match the name of the directory that
contains all relevant FITS files.\\

\texttt{--useSE} ~~$-$ \hangindent=2.7cm option to use \textit{SExtractor} 
to detect sources in each image rather than \An's bundled ``images2xy'' 
program. \\

\noindent Solved files are renamed \texttt{<base>\_solved.fits}. These 
are the original FITS images with headers updated to contain the WCS
solutions. 
The original FITS image as well as all of \An's auxiliary
output files are moved to a directory called \texttt{astrometric\_solns}.
See Appendix \ref{Anoutputs} for a list of the output files. \\

\noindent \An~software can take a long time if it must run 
through every available index file in order to find one that covers the
image. In order to expedite the process, \texttt{astrometry.py} tells \An~to:
\begin{itemize}
    \item skip running the FITS files through a ``sanitizer'', which cleans up
    non-standards-compliant images
    \item query only index files within 1$^{\circ}$ of the RA and Dec in the 
    image headers
    \item expect image pixel scales in the range $0.1 - 0.45$''/pix 
\end{itemize}
The range in the last step is enough to cover both binned and unbinned images
and ensures that only index files of the proper scale are used in 
determining the fit. (This step probably does not affect much as
that's a rather large range.) \\

\subsection{\texttt{fix\_astrometry.py}}
The \An~software does a great job with the astrometry near the 
center of the chip. Most WISP fields
were observed on the top $1/3$ of the chip. At the bottom and edges of 
the images, the WCS gets further off. 
\texttt{fix\_astrometry.py} uses IRAF's CCMAP task to calculate
a second order solution that takes care of the small 
distortions and rotations at the
edges. \\

\texttt{usage: fix\_astrometry.py [-h] wispfield} \\

\texttt{wispfield} $-$ \hangindent=2.7cm the WISP parallel field whose images
are to be fit with improved WCS solutions. This must match the name of 
the directory that contains all relevant FITS files.\\

\texttt{Required files}: \\
\indent \indent \texttt{result.fits} $-$ the SDSS
catalog\footnote{Download the catalog from 
\url{http://skyserver.sdss3.org/dr10/en/tools/search/radial.aspx} in fits
format} for the WISP field. \\

\noindent \texttt{fix\_astrometry.py} first runs \textit{SExtractor}
on all images and outputs a catalog and segmentation map named
\texttt{<base>\_astr\_cat.fits} and \texttt{<base>\_astr\_seg.fits},
respectively. For each image, it then matches the sources detected in the
Palomar image with the SDSS catalog. A file to be used with CCMAP, 
\texttt{SDSS.match}, is created
listing the Palomar $x,y$ coordinates with the SDSS RA and Dec for each 
source. Finally, it launches CCMAP, which allows the user to interactively
delete obvious residuals in the astrometric solution. The 
\texttt{<base>\_solved.fits} headers are updated with the new and 
improved WCS. 

\subsection{\texttt{combine.py}}
Once all images have corrected WCS solutions, images from like filters
are stacked using IRAF's IMCOMBINE task. Before combining, the user should
inspect all images and remove any for which the seeing is significantly worse.
Use IRAF's IMEXAM to check the FWHM of a few point sources across each image.
Make sure to check the observing logs as well for any mention of clouds or 
other problems. \texttt{combine.py} will prompt the user to stop and 
perform this inspection if they haven't already. \\

\texttt{usage: combine.py [-h] [--rejection {none,minmax,ccdclip,crreject,sigclip,avsigclip,pclip}] \\
\indent \indent \indent \indent \indent \texttt{[--combine {average,median,sum}] wispfield}} \\

\texttt{wispfield} $-$ \hangindent=2.7cm the WISP parallel field for which 
to stack image. This must match the name of the directory that contains all
relevant FITS files.\\

\texttt{--rejection} $-$ \hangindent=2.7cm the rejection algorithm to use. 
Default is \texttt{crreject}, which rejects only positive pixels and is 
appropriate for cosmic ray rejection. \\

\texttt{--combine} ~~$-$ \hangindent=2.7cm the type of combining operation 
to perform. Default is \texttt{median}. \\

\noindent See the IMCOMBINE help page for more information about parameter 
options.

\noindent For each filter, \texttt{combine.py} determines which images
need to be combined and prepares all parameters and files for use with 
IMCOMBINE. Multiplicative scale factors are calculated for each image
from the airmass (\texttt{[wispfield]\_[filter].scale}). 
The scale factor $=10^{0.4 ~ \kappa_F~ X}$, where $\kappa_F$ is the airmass 
coefficient calculated for the Palomar site\footnote{Taken from 
\url{http://www.ifa.hawaii.edu/~rgal/science/dposs/ccdproc/ccdproc_node3.html}, 
where $\kappa_g = -0.152$, $\kappa_r=-0.094$, and $\kappa_i=-0.07$.} 
and filter $F$.
Additive zero level shifts are calculated from the median value for 
each image. All zero corrections are done with respect to the first image,
so the images are sorted such that the lowest sky is listed first.  \\

\noindent The readnoise is necessary for IMCOMBINE's sigma clipping routines,
and can be estimated by binning the counts from an averaged bias frame. The
mean of the resulting histogram is the bias offset and 
the width of the distribution is $\sigma_{ADU} =~$readnoise/gain$~=~$FWHM.
The user should include in each WISP field directory the master bias from
each night the field was observed, and \texttt{combine.py} will estimate
an average readnoise to use for all images.\\

\noindent Finally \texttt{combine.py} reports the number of images available
in each filter and asks the user to choose the number of low/high pixels
to reject or the number of pixels to keep 
(depending on the rejection algorithm used). \\

\noindent IMCOMBINE uses the WCS in the headers to calculate and apply 
offsets to each image so they are properly aligned before stacking. 
Combined images are called \texttt{[wispfield]\_[filter].fits}. 
Sigma images are also created (\texttt{[wispfield]\_[filter]\_sig.fits}),
and the log file for the combination is called
\texttt{[wispfield]\_imcombine.log}. \\

\subsection{\texttt{align\_images.py}}
If there are 2 filters observed that night, \texttt{combine.py} will 
import \texttt{align\_images.py} to align those images using 
the WCS in the headers and IRAF's 
task WREGISTER.

\vspace{4 mm}
\section{Calibration}
For each field, the Palomar photometry is calibrated against the SDSS catalog 
(\texttt{result.fits}). \texttt{calibrate\_sdss.py} calculates both a zero 
point shift and a color term and saves them for use in calibrating the final 
catalog. \\

\texttt{usage: calibrate\_sdss.py [-h] wispfield} \\

\texttt{wispfield} $-$ \hangindent=2.7cm the WISP parallel field for which
to calibrate photometry. This must match the name of the directory that
contains all relevant FITS files.\\


\noindent \textit{SExtractor} is run in dual image mode
on the combined Palomar images with 
$1\sigma$ detection and analysis thresholds and a minimum area for 
detection of 5 pixels. Such low thresholds are used to ensure that as much
light as possible is included in each source's automatic magnitude. The 
Palomar catalogs are then match with the SDSS catalog using a $0.5''$
matching threshold. \\

\noindent The zero point shift, $zp = m_{SDSS} - m_{Palomar}$ 
is calculated for all sources, and those outside of $68\%$ of the
mean $zp$ are excluded. There is a residual slope visible in a plot of 
$zp$ that depends on color (top two rows of Fig.~\ref{fig:cals}).
This dependence is especially prevalent in the $g$ band.
Color terms to correct for this dependence 
are determined by fitting a line to the array of 
zero point shifts. Outliers (plotted in red) that are too far from the 
line in the 
$y$-direction are removed, and a new line (plotted in black) 
is fit to the remaining points. \\


\noindent where $\alpha_0$ is the slope of the best-fit line (the color term)
and $\alpha_1$ is the offset (the zero point shift). Several plots
are then created to check the quality of the calibration (rows $3-4$ in 
Fig.~\ref{fig:cals}). The last row shows the residuals of the calibrated 
data. For perfectly calibrated photometry, there would be no remaining
color dependence and the solid line would have a slope of zero. 
The Palomar photometry is then calibrated by:
\begin{equation}
m_{calibrated} = m_{Palomar} + \alpha_0~(g-i)_{Palomar} + \alpha_1,
\label{eqn:cal}
\end{equation}

\subsection{1 Palomar Filter, 1 WFC3 filter}
Fields that were observed with only one SDSS filter require a second
filter for determination of the color terms. For this we use the WFC3
UVIS filters that are available for the field. Possible color combinations
include ($g-$F814W), (F606W$-i$), ($g-$F600LP), and (F475X$-i$).
Use \texttt{calibrate\_sdss\_hst.py} in place of 
\texttt{calibrate\_sdss.py}.


\vspace{4 mm}
\section{Final Catalog}
If both filters are present, SE is run in dual image
mode with the $i$ band used for detection.  SE parameters are listed
in Table \ref{tab:SE}. SE is run with a zero point magnitude of $0.0$,
so that the zero point calculated in \S \ref{sec:cal} can be applied
to the photometry. \\
\begin{table}
\begin{center}
\caption{\textit{SExtractor} Parameters}
\label{tab:SE}
\begin{tabular}{cc}
\hline
Parameter & Value \\
\hline
\texttt{DETECT\_MINAREA} & \texttt{5} \\
\texttt{THRESH\_TYPE} & \texttt{RELATIVE} \\
\texttt{DETECT\_THRESH} & \texttt{2.2} \\
\texttt{ANALYSIS\_THRESH} & \texttt{2.2} \\
\texttt{DEBLEND\_NTHRESH} & \texttt{32} \\
\texttt{DEBLEND\_MINCONT} & \texttt{0.005} \\
\hline
\texttt{WEIGHT\_TYPE} & \texttt{None} \\
\hline
\texttt{PHOT\_APERTURES} & \texttt{30} \\
\texttt{PHOT\_AUTOPARAMS} & \texttt{2.0, 3.5} \\
\texttt{PHOT\_AUTOAPERS} & \texttt{0.0, 0.0} \\
\texttt{GAIN} & \texttt{GAIN} $\times$ (total exptime) \\
\hline
\texttt{BACK\_SIZE} & \texttt{64} \\
\texttt{BACK\_FILTERSIZE} & \texttt{3} \\
\texttt{BACKPHOTO\_TYPE} & \texttt{LOCAL} \\
\texttt{BACKPHOTO\_FILTERSIZE} & \texttt{24} \\
\hline
\end{tabular}
\end{center}
\end{table}

\noindent \texttt{AUTO} magnitudes are calibrated according
to eqn. (\ref{eqn:cal}). We do not perform an aperture correction on the
\texttt{AUTO} photometry. We checked the curve of growth for a variety of
sources and could not determine a single correction factor for all 
types of sources. The percentage of light enclosed by the Kron radius
depends on the Sersic index. (See, for example, 
\htmladdnormallink{Graham \& Driver 
(2005)}{http://adsabs.harvard.edu/abs/2005PASA...22..118G}.) \\

The Palomar and WISP catalogs (fin\_F110.cat, fin\_F140.cat, fin\_F160.cat)
are matched with a matching threshold of $0\farcs75$. For some fields, 
the astrometry between the WISP and Palomar catalogs is off by more then 
$0\farcs75$ and 2.0" is used instead. In these cases, the field's 
README mentions is. 


\texttt{make\_final\_catalog.py} runs \textit{SExtractor} a final time
with detection and analysis thresholds of $2.2\sigma$ so that all sources
are detected at $\geq5\sigma$. The zero point shift and color terms 
calculated by \texttt{calibrate\_sdss.py} are applied to all Palomar
photometry. Palomar sources are matched to the WISP catalog with a 0.5''
matching radius. The final catalog includes both Palomar and WISP
RA and Dec and photometry from all available filters.\\

\texttt{usage: make\_final\_catalog.py [-h] wispfield} \\

\texttt{wispfield} $-$ \hangindent=2.7cm the WISP parallel field for which
to construct the final catalog. This must match the name of the directory that
contains all relevant FITS files.\\


\end{document}
