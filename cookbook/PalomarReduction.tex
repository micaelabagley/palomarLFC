\documentclass{article}
\usepackage{graphicx}
\usepackage{vmargin}
\usepackage{hyperref}
\usepackage{amsmath}
\usepackage{url}
\setpapersize{A4}
\setmarginsrb{2.5cm}{2.5cm}{2.4cm}{2.8cm}{12pt}{14pt}{12pt}{22pt}
\pagestyle{empty}

%%% define \maketitle without the vertical space %%%
\makeatletter
\def\@maketitle{%
  \newpage
%  \null% DELETED
%  \vskip 2em% DELETED
  \begin{center}%
  \let \footnote \thanks
    {\LARGE \@title \par}%
    \vskip 0.75em%
    {\large
      \lineskip .5em%
      \begin{tabular}[t]{c}%
        \@author
      \end{tabular}\par}%
    \vskip 0.75em%
    {\small \@date}%
  \end{center}%
  \par
  \vskip 1em}
\makeatother
%%% reduce spacing between items in lists %%%
\newlength{\wideitemsep}
\setlength{\wideitemsep}{.5\itemsep}
\addtolength{\wideitemsep}{-7pt}
\let\olditem\item
\renewcommand{\item}{\setlength{\itemsep}{\wideitemsep}\olditem}
%%% reduce size of $\sim$ for use with bash paths %%%
\newcommand{\ttsim}{\raise.17ex\hbox{$\scriptstyle\mathtt{\sim}$}}
%%% define how bash code will be displayed (indented, preceded by '#' %%%
\newcommand{\shellcmd}[1]{\indent\indent\texttt{\small\# #1}\\}
%%%
\newcommand{\An}{\textit{Astrometry.net}}

\title{Palomar 200" LFC Reduction README}
%\author{Micaela Bagley}
\author{February 2015}
\date{}

\begin{document}
\maketitle

\section{Reduction}
Reduction is performed in the usual way. Very few observation runs 
included darks, and most flats taken were dome flats. For consistency, 
we do not perform dark subtraction. Additionally, all flat fielding is done 
with dome flats, even if twilight flats were taken on a given night.

%\vspace{3 mm}
\section{Astrometry, Alignment \% Stacking}
The astrometric solution for each field is found through two steps:
(1) The \textit{Astrometry.net} software returns a rough 
(though very good) WCS solution, and (2) IRAF's CCMAP task calculates
a second order solution that hopefuly takes care of the small distortions
and rotations at the chip's edges. We use SDSS astrometry as a reference.
Astrometry was by far the most challenging
step, and it is not perfect in all fields.\\

\noindent \textbf{NOTE!} The astrometry is best on the chip's top half. 
The astrometric solution is very poor near the bottom of the chip
and can have problems very close to the chip edges. In some fields, 
the Palomar astrometry is significantly different from that in the 
WISP catalogs.
\textbf{Be careful when matching Palomar and WISP catalogs. You may
be missing many valid matches with a small matching radius.} 
Compare WISP.reg and Palomar.reg to check. \\

\noindent Images are stacked using IRAF's IMCOMBINE task and the 
crreject rejection algorithm. 
We only include images 
with perfectly consistent astrometry, at least on the upper half of the chip.
We throw out images with (1) very poor seeing, (2) abnormally elongated 
sources, or (3) very few sources / 
an unusually high sky background due to clouds. \\

\noindent Multiplicative scale factors are calculated for each image
from the airmass. 
The scale factor $=10^{0.4 ~ \kappa_F~ X}$, where $\kappa_F$ is the airmass 
coefficient calculated for the Palomar site\footnote{Taken from 
\url{http://www.ifa.hawaii.edu/~rgal/science/dposs/ccdproc/ccdproc\_node3.html},
where $\kappa_g = -0.152$, $\kappa_r=-0.094$, and $\kappa_i=-0.07$.} 
and filter $F$.
Additive zero level shifts are calculated from the median value for 
each image. 
The readnoise, necessary for crreject,
is estimated by binning the counts from an averaged bias frame. The
mean of the resulting histogram is the bias offset and 
the width of the distribution is $\sigma_{ADU} =~$readnoise/gain$~=~$FWHM.\\

\noindent If there are 2 filters observed that night, the images are 
aligned using the WCS in the headers and IRAF's task WREGISTER.

%\vspace{2 mm}
\section{Calibration}\label{sec:cal}
For each field, the Palomar photometry is calibrated against the SDSS catalog. 
\textit{SExtractor} is run in dual image mode
on the combined Palomar images with 
$1\sigma$ detection and analysis thresholds. 
Such low thresholds are used to ensure that as much
light as possible is included in each source's automatic magnitude. The 
Palomar catalogs are then matched with the SDSS catalog using a $1''$
matching threshold. We consider only SDSS sources with S/N$\geq10$ in 
both $g$ and $i$. \\

\noindent Ater removing outliers, we calculate a zero point shift and -- 
for fields
observed with both $g$ and $i$ -- a color term
using the instrumental Palomar colors. 
The Palomar photometry is then calibrated by:
\begin{equation}
m_{calibrated} = m_{Palomar} + \alpha~(g-i)_{Palomar} + zp,
\label{eqn:cal}
\end{equation}
where $\alpha$ is the color term
and $zp = m_{SDSS} - m_{Palomar}$ is the zero point shift. Several plots
are created to check the quality of the calibration.
The last row shows the residuals of the calibrated 
data. For perfectly calibrated photometry, there would be no remaining
color dependence and the solid line would have a slope of zero. \\

%\vspace{2 mm}
\section{Final Catalog}
If both filters are present, SE is run in dual image
mode with the $i$ band used for detection.  SE parameters are listed
in Table \ref{tab:SE}. SE is run with a zero point magnitude of $0.0$,
so that the zero point calculated in \S \ref{sec:cal} can be applied
to the photometry. \\
\begin{table}
\begin{center}
\caption{\textit{SExtractor} Parameters}
\label{tab:SE}
\begin{tabular}{cc}
\hline
Parameter & Value \\
\hline
\texttt{DETECT\_MINAREA} & \texttt{5} \\
\texttt{THRESH\_TYPE} & \texttt{RELATIVE} \\
\texttt{DETECT\_THRESH} & \texttt{2.2} \\
\texttt{ANALYSIS\_THRESH} & \texttt{2.2} \\
\texttt{DEBLEND\_NTHRESH} & \texttt{32} \\
\texttt{DEBLEND\_MINCONT} & \texttt{0.005} \\
\hline
\texttt{WEIGHT\_TYPE} & \texttt{None} \\
\hline
\texttt{PHOT\_APERTURES} & \texttt{30} \\
\texttt{PHOT\_AUTOPARAMS} & \texttt{2.0, 3.5} \\
\texttt{PHOT\_AUTOAPERS} & \texttt{0.0, 0.0} \\
\texttt{GAIN} & \texttt{GAIN} $\times$ (total exptime) \\
\hline
\texttt{BACK\_SIZE} & \texttt{64} \\
\texttt{BACK\_FILTERSIZE} & \texttt{3} \\
\texttt{BACKPHOTO\_TYPE} & \texttt{LOCAL} \\
\texttt{BACKPHOTO\_FILTERSIZE} & \texttt{24} \\
\hline
\end{tabular}
\end{center}
\end{table}

\noindent \texttt{AUTO} magnitudes are calibrated according
to eqn. (\ref{eqn:cal}). We do not perform an aperture correction on the
\texttt{AUTO} photometry. We checked the curve of growth for a variety of
sources and could not determine a single correction factor for all 
types of sources. The percentage of light enclosed by the Kron radius
depends on the Sersic index. (See, for example, 
\htmladdnormallink{Graham \& Driver 
(2005)}{http://adsabs.harvard.edu/abs/2005PASA...22..118G}.) \\

\noindent The limiting magnitude is determined for each filter by binning the 
flux calculated in 2.5" circular apertures placed randomly on the sky
in the image. The distribution of fluxes is fit with a Gaussian and 
the $\sigma$ is calculated. \textbf{All fluxes fainter than the
$1\sigma$ limit are set to the $1\sigma$ magnitude,} and their uncertainties
are set to $0.0$.  \\

\noindent The Palomar and WISP catalogs 
(fin\_F110.cat, fin\_F140.cat, fin\_F160.cat)
are matched with a matching threshold of $0.75$". For some fields, 
the astrometry between the WISP and Palomar catalogs is off by more then 
$0.75$" and 2.0" is used instead. This is indicated by a note in the
field's README. In all cases, the images were visually inspected 
to confirm the quality of the matches. \\

\noindent In both output catalogs, photometry for filters that do not exist 
is set
to $99.99$ with $-9.99$ uncertainties. This is also true for the Palomar
RA and Dec of sources in the WISP catalog there were not matched to a 
Palomar source.

\section{Final Products}

\begin{description}
%\parshape 2 0cm \linewidth \listindent \dimexpr\linewidth-\listindent\relax
  \item[Palomar.cat] Catalog of calibrated photometry. Only sources from 
                     the full overlap region of the combined image 
                     are included. Sources near the edges, where fewer
                     input images are contributing to the combined stack, 
                     are excluded. \\
  \item[Palomar.reg] Region file of sources in Palomar.cat \\
  \item[Palomar\_WISPS.cat] WISP catalog with matched Palomar sources. All
                     NIR photometry available from the HST reduction is 
                     included.\\
  \item[README.txt] Info on observations, reductions, stacking, and calibration
                     for the given field. \\
  \item[result.fits] SDSS catalog \\
  \item[sdss\_calibration.dat] Calibration information: color terms, zero 
                               points, and limiting magnitudes for each 
                               filter. \\
  \item[sdss\_calibration.pdf] Calibration plots showing the determination of 
                     the color terms, a comparison of the calibrated 
                     photometry with SDSS, and the residual slopes 
                     post-calibration. \\
  \item[(wispfield)\_g\_final\_cat.fits] Full, uncalibrated $g$ band SE
                                         catalog \\
  \item[(wispfield)\_g\_final\_seg.fits] SE segmentation map, present only 
                                    if $g$ is the only filter present.
                                    If the field was observed with $i$ as well,
                                    the images were run in dual image mode
                                    and the $i$ band segmentation map is
                                    the only one produced. \\
  \item[(wispfield)\_g.fits] Combined $g$ image, full chip \\
  \item[(wispfield)\_g\_WISPFOV.fits] Combined $g$ image, centered at 
                                      WISP RA,Dec and trimmed to $3'\times3'$\\
  \item[(wispfield)\_i\_final\_cat.fits] Full, uncalibrated $i$ band SE
                                         catalog \\
  \item[(wispfield)\_i\_final\_seg.fits] SE segmentation map \\
  \item[(wispfield)\_i.fits] Combined $i$ image, full chip \\
  \item[(wispfield)\_i\_WISPFOV.fits] Combined $i$ image, centered at 
                                      WISP RA,Dec and trimmed to $3'\times3'$\\
  \item[WISP.reg] Region file of WISP catalog. Sources matched to Palomar
                  are indicated by thick circles
\end{description}


\end{document}
