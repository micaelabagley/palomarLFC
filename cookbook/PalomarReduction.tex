\documentclass{article}
\usepackage{graphicx}
\usepackage{vmargin}
\usepackage{cancel}
\usepackage{hyperref}
\setpapersize{A4}
\setmarginsrb{2.5cm}{2.5cm}{2.4cm}{2.8cm}{12pt}{14pt}{12pt}{22pt}
\pagestyle{empty}

%%% define \maketitle without the vertical space %%%
\makeatletter
\def\@maketitle{%
  \newpage
%  \null% DELETED
%  \vskip 2em% DELETED
  \begin{center}%
  \let \footnote \thanks
    {\LARGE \@title \par}%
    \vskip 0.75em%
    {\large
      \lineskip .5em%
      \begin{tabular}[t]{c}%
        \@author
      \end{tabular}\par}%
    \vskip 0.75em%
    {\small \@date}%
  \end{center}%
  \par
  \vskip 1.5em}
\makeatother

\newlength{\wideitemsep}
\setlength{\wideitemsep}{.5\itemsep}
\addtolength{\wideitemsep}{-7pt}
\let\olditem\item
\renewcommand{\item}{\setlength{\itemsep}{\wideitemsep}\olditem}

\newcommand{\ttsim}{\raise.17ex\hbox{$\scriptstyle\mathtt{\sim}$}}

\newcommand{\shellcmd}[1]{\indent\indent\texttt{\footnotesize\# #1}\\}

\title{Palomar 200" LFC Reduction Cookbook}
\author{Micaela Bagley}
\date{December 2014}

\begin{document}
\maketitle

\section{Introduction}

\section{Requirements}
\begin{itemize}
\item \texttt{numpy
\item pyfits
\item pyraf
\item astropy
\item scipy
\item matplotlib
\item astrometry.net} (see below)
\end{itemize}
pip install -r requirements.txt

\section{Reduction}

\section{Astrometry \& Alignment}
\begin{enumerate}
\item \texttt{astrometry.py}
\item \texttt{fix\_astrometry.py}
\item \texttt{combine.py}
\item \texttt{align\_images.py}
\item (\texttt{imalign.pro})
\end{enumerate}
\subsection{\texttt{astrometry.py}}
Get a rough (though very good) approximate WCS solution for each image
in a WISP parallel field using the Astrometry.net software package.
Uses the RA,Dec from the headers as an initial guess for each image.
Set \texttt{--useSE} to use SourceExtractor to detect sources rather than
Astrometry.net's bundled \texttt{images2xy} program. (Note: if SExtractor
is used, the xyls output files may be incorrect.)

Optional arguments sent to Astrometry.net software:
--no-plots
--no-fits2fits
--scale-units
--crpix-center
--radius
--ra 
--dec
--new-fits




\section{Calibration}
asdf

\appendix
\section{Installing Astrometry.net}
Designate a directory for installation of Astrometry.net and its requirements.
In the following example, the installation directory is
\texttt{/home/bagley/Software}. \\

\noindent (Thanks to Michael Gordon (UMN-MIfA) for these installation 
instructions, which are also provided as a text file (\texttt{astro\_setup.txt})

\subsection{CFITSIO}
CFITSIO is a library of C and Fortran subroutines for reading and writing
data files in FITS data format, available from NASA's High Energy
Astrophysics Science Archive Research Center (HEASARC). 

\shellcmd{mkdir \ttsim/Software/cfitsio}
\shellcmd{cd \ttsim/Software/cfitsio}
\shellcmd{wget ftp://heasarc.gsfc.nasa.gov/software/fitsio/c/cfitsio\_latest.tar.gz}
\shellcmd{tar xzvf cfitsio\_latest.tar.gz}
\shellcmd{cd cfitsio}
\shellcmd{./configure --prefix=/home/bagley/Software/cfitsio}
\shellcmd{make}
\shellcmd{make install}

\noindent Add the following to your \ttsim/.bashrc and source it: \\
\shellcmd{export PKG\_CONFIG\_PATH=/home/bagley/Software/cfitsio/lib/pkgconfig}

\noindent Run the following commands, if they print something, you're good: \\
\shellcmd{pkg-config --cflags cfitsio}
\shellcmd{pkg-config --libs cfitsio}

\subsection{Astrometry.net}
If you are running Python in a Virtual Environment, make sure you are 
in your virtualenv before completing the next steps. \\
\shellcmd{mkdir \ttsim/Software/astrometry}
\shellcmd{cd \ttsim/Software/astrometry}
\shellcmd{wget http://astrometry.net/downloads/astrometry.net-0.46.tar.bz2}
\shellcmd{tar xjvf astrometry.net-0.46.tar.bz2}
\shellcmd{cd astrometry.net-0.46}
\shellcmd{make}
\shellcmd{make py}
\shellcmd{make extra}
\shellcmd{make install INSTALL\_DIR=/home/bagley/Software/astrometry}

\noindent Add the following to your \ttsim/.bashrc and source it: \\
\shellcmd{export PATH="\$PATH:/home/bagley/Software/astrometry/bin"} 

\noindent In your virtualenv, run astrometry.net as: \\
\shellcmd{solve-field --no-plot img.fits .....}

\subsection{Index Files}
Astrometry.net requires index files, processed from an astrometric reference
catalog such as USNO-B1 or 2MASS. Pre-cooked index files built from the 
2MASS catalog are available \htmladdnormallink{here}
{http://data.astrometry.net/4200}. Use the \texttt{wget} script to download
the full catalog of index files, requiring about 10G of space. Alternatively, 
use the healpix png's to determine in which tiles your fields of interest
reside and download only the relevant index files. \\

\noindent The software expects index files to be in 
\texttt{\ttsim/Software/astrometry/data}. If you don't have enough space
there, symlimk the \texttt{data} directory to their actual location. \\
\shellcmd{cd \ttsim/Software/astrometry}
\shellcmd{rmdir data}
\shellcmd{ln -s /other/thing/data data}



\end{document}
